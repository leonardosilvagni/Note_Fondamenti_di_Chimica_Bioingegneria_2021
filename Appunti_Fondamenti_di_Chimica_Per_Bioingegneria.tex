\documentclass{article}
\usepackage[italian]{babel}
\usepackage[letterpaper,top=2cm,bottom=2cm,left=3cm,right=3cm,marginparwidth=1.75cm]{geometry}
\usepackage{amsmath}
\usepackage{graphicx}
\usepackage{float}
\usepackage[colorlinks=true, allcolors=blue]{hyperref}

\title{Appunti Fondamenti di chimica per Bioingegneria}
\author{Leonardo Silvagni}

\begin{document}
\maketitle

\begin{abstract}

\end{abstract}

\section{La Materia}
La materia è composta da atomi. Fino a fine 1800 si pensava queste fossero le particelle fondamentali ed indivisibili (da cui a-tomos, indivisibile) della materia.  
Solo successivi e sempre più avanzati studi sull'elettricità hanno portato alla scoperta di varie particelle subatomiche
\subsection{Caratteristiche principali delle particelle subatomiche}

\begin{table}[H]
\centering
\begin{tabular}{c|c|c|c|c}
Particella & Simbolo & Massa & Carica & Anno scoperta \\\hline
elettrone & $e^-$ & $9,109\dot 10^{-28}g$ & $-1,602\dot 10^{-19}C$ & 1897 \\
protone & $p^+$ & $1,67\dot 10^{-24}g$ & $1,602\dot 10^{-19}C$ & 1917 \\
neutrone & $n$ & $1,67\dot 10^{-24}g$ & $0$ & 1932 
\end{tabular}
\caption{\label{tab:widgets}Caratteristiche principali delle particelle subatomiche.}
\end{table}
\subsection{I Modelli Atomici}
\begin{enumerate}
\item Dalton    1803
\item Thomson   1904 A panettone
\item Rhuterford    1911-1917 Gli esperimenti finali portarono alla scoperta del protone
\item Bohr - Sommerfeld     1913
\item Shrodinger    1926
\end{enumerate}
E' da notare che Bohr studiò solo l'atomo di Idrogeno, Sommerfeld estese il modello a specie chimiche. La novità principale di questo modello è stata l'introduzione delle orbite circolari ed ellittiche degli elettroni attorno al nucleo. 
In un atomo neutro, protoni ed elettroni si trovano nella stessa quantità, mentre i neutroni sono in numero variabile.
\subsection{Il nucleo}
Protoni, neutroni e reazioni nucleari non sono di interesse della chimica, in quanto nelle reazioni chimiche il nucleo rimane intoccato, reazioni chimiche sono relazioni tra elettroni. 
\begin{itemize}
    \item \textbf{NUCLEONI} $\rightarrow$ Protoni e neutroni
    \item Somma del numero di protoni e neutroni, numero di massa $A$
    \item Numero di protoni $Z$
    \item Numero di neutroni $A-Z$
\end{itemize}
\subsubsection*{Nota}
Specie chimiche con lo stesso numero di elettroni danno le stesse reazioni chimiche e possiedono le stesse proprietà chimico fisiche
\subsubsection{I NUCLIDI}
I nuclidi sono le diverse specie di atomi caratterizzate da nuclei di definita composizione Un nuclide viene rappresentato dal simbolo che indica la particella, con apice e pedice a sinistra per indicare numero di massa e numero atomico, apice e pedice a destra per indicare la carica e il numero di atomi
\[
^A_ZX^\pm _n
\]
Due nuclidi si dicono \textbf{isobari} se han diverso numero atomico ma ugual numero di massa
\[
^{14}_6C \qquad  ^{14}_7N
\]
Due nuclidi si dicono isotopi se han lo stesso numero atomico ma diverso numero di massa. Cambia il numero di neutroni
\[
^{16}_8O \qquad ^{17}_8O
\]
\subsubsection{Gli isotopi}
Tutti gli elementi chimici presenti in natura sono costituiti da miscele di \textbf{nuclidi diversi},
caratterizzati tutti dallo stesso numero atomico ma diverso numero di massa, isotopi appunto.\\
\[
^{20}_{10}Ne \qquad ^{21}_{10}Ne \qquad^{22}_{10}Ne 
\]
Esempi di isotopi di neon
\[
^{16}_{8}O \ p= \ 99.979\% \qquad ^{17}_{8}O \ p= \ 0.037\%  \qquad^{18}_{8}O \ p= \ 0.204\%  
\]
Isotopi naturali dell'ossigeno, non radioattivi\\
Gli isotopi naturali sono presenti in natura in forma stabile, non sono soggetti a decadimento radioattivo.
La \textit{abbondanza percentuale} è intesa come numero di atomi, e sono costanti nell'universo, salvo ambienti specifici dove possono casualmente esserci variazioni minime dei valori
\[
^{1}_{1}H \ Prozio \ p= \ 99.984\% \qquad ^{2}_{1}H \ Deuterio\ p= \ 0.016\% 
 \qquad^{3}_{1}H \ Trizio \ p= \ 10^{-15}\% \]
Il prozio è notato anche semplicemente con Idrogeno, il deuterio con Idrogeno pesante e si nota anche solo con $D$, o $^2_1D$, Il Trizio con $T$ o $^3_1T$\\
Se noi sostituissimo all'idrogeno il deuterio per formare l'acqua, per formare l'acqua pesante $^2_1H_2O$, questo elemento bolle a $101.43C$, e fonde a $3.82C$.\\
\subsection{La massa}
\subsubsection{La massa atomica}
Le masse degli atomi vengono determinate con uno strumento che si chiama spettrometro di massa e sono masse atomiche \textbf{assolute},
i loro pesi sono molto piccoli.
E' stato perciò conveniente definire una particolare unità di misura, l'unità di massa atomica (uma), definita come \\
1 uma = $\frac{1}{12}$ della massa di $^{12}_6C$ = $1.6606 x 10^{-24}g$\\
Dividendo la masssa atomica assoluta per l'unità di massa atomica posso ottenere la massa atomica relativa.
\[m_{ar}= \frac{m_{aa}}{uma}\]
\subsubsection{La massa di un elemento}
\qquad La massa di un elemento, ovvero la massa atomica media o il peso atomico.\\
Abbiamo visto che gli elementi sono composti da miscele di diversi isotopi. Nei calcoli chimici
dobbiamo considerare la presenza di ogni isotopo per ottenere il valore medio della massa atomica di un 
elemento. Indicando con "E" un generico elemento della tavola periodica, avremo 
\begin{equation}\label{massa_atomica}M_e= \frac{\Sigma _i M_i \ p_i}{100}\end{equation}
$M_i$ è calcolato in uma.
Esempio:
\[M_{^{16}_8O}=15.9949 \ uma \]
\[M_{^{17}_8O}=16.9991 \ uma \]
\[M_{^{18}_8O}=17.9992 \ uma \]
Da cui, utilizzando la formula \eqref{massa_atomica}, si ottiene la massa media dell'ossigeno
\[M_{O}=15.9994 \ uma \]
\subsubsection*{Nota}
Conoscendo la formula chimica e sommando le masse degli atomi si ottiene la massa molecolare.
\subsubsection{La mole ed il Numero di Avogadro}
La maggior parte delle reazioni chimiche coinvolge un numero enorme di particelle(atomi, molecole etc...).
E' stato perciò conveniente definire una nuova grandezza, la mole (mol), che rappresenta un numero grande e 
fisso di particelle, comparabile alla quantità di particelle usate in un esperimento reale.\\
DEF: \textit{Una mole di una determinata sostanza è la quantità di tale sostanza che contiene tante
entità elementari (Atomi, ioni, molecole etc...) quanti sono gli atomi contenuti in 12 g esatti di $^{12}_6C$.}\\ 
Possiamo determinare facilmente questo numero di atomi: basta dividere la massa totale per la massa di un atomo.
Da questo il numero di Avogadro è
\\ $N_A = 6.022\cdot 10^{23}$ entità/mol.\\
\subsubsection{La massa molare}
Esempio : \textit{Si voglia determinare la massa molare di $H_2O$}
\begin{enumerate}
    \item La massa espressa in uma di $H_2O$ è di 18.015 uma  
    \item La massa in grammi di una molecola di $H_2O$ è 18.015 uma  $\cdot 1.6606\cdot 10^{-24}$
    \item In una mole di $H_2O$ ci sono $N_A$ molecole
    \item La massa espressa in grammi di una mole di $H_2O$ è (numero molecole)$\cdot$(massa in g di una molecola)
\end{enumerate}
Una mole di $H_2O$ pesa 18.015 g, così per qualsiasi sostanza, il peso di una mole di un dato elemento X
è pari allo stesso valore numerico della sua massa espressa in uma.



\end{document}