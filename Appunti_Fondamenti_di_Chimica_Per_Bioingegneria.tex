\documentclass{article}
\usepackage[italian]{babel}
\usepackage[letterpaper,top=2cm,bottom=2cm,left=3cm,right=3cm,marginparwidth=1.75cm]{geometry}
\usepackage{amsmath}
\usepackage{graphicx}
\usepackage{float}
\usepackage[colorlinks=true, allcolors=blue]{hyperref}
%\usepackage{chemmacros}
%\usepackage{chemgreek}
%\usechemmodule{redox}
%\chemsetup[redox]{pos=top}
%\ox{+1,Na}, \ox{2,Ca}, \ox{-2,S}, \ox{-1,F}

\title{Appunti Fondamenti di chimica per Bioingegneria}
\author{Leonardo Silvagni}

\begin{document}
\maketitle

\begin{abstract}
Appunti delle lezioni di Fondamenti di chimica per Bioingegneria, la parte di chimica
inorganica, tenuta dall'inimitabile professor Mozzon. 
\end{abstract}

\section{La Materia}
%\ox{+1,Na}, \ox{2,Ca}, \ox{-2,S}, \ox{-1,F} 
La materia è composta da atomi. Fino a fine 1800 si pensava queste fossero le particelle fondamentali ed indivisibili (da cui a-tomos, indivisibile) della materia. 
Solo successivi e sempre più avanzati studi sull'elettricità hanno portato alla scoperta di varie particelle subatomiche
\subsection{Caratteristiche principali delle particelle subatomiche}
\begin{table}[H]
\centering
\begin{tabular}{c|c|c|c|c}
Particella & Simbolo & Massa & Carica & Anno scoperta \\\hline
elettrone & $e^-$ & $9,109\dot 10^{-28}g$ & $-1,602\dot 10^{-19}C$ & 1897 \\
protone & $p^+$ & $1,67\dot 10^{-24}g$ & $1,602\dot 10^{-19}C$ & 1917 \\
neutrone & $n$ & $1,67\dot 10^{-24}g$ & $0$ & 1932 
\end{tabular}
\caption{\label{tab:widgets}Caratteristiche principali delle particelle subatomiche.}
\end{table}
\subsection{I Modelli Atomici}
\begin{enumerate}
\item Dalton    1803
\item Thomson   1904 A panettone
\item Rhuterford    1911-1917 Gli esperimenti finali portarono alla scoperta del protone
\item Bohr - Sommerfeld     1913
\item Shrodinger    1926
\end{enumerate}
E' da notare che Bohr studiò solo l'atomo di Idrogeno, Sommerfeld estese il modello ad ulteriori specie chimiche. La novità principale di questo modello è stata l'introduzione delle orbite circolari ed ellittiche degli elettroni attorno al nucleo. 
In un atomo neutro, protoni ed elettroni si trovano nella stessa quantità, mentre i neutroni sono in numero variabile.
\subsection{Il nucleo}
Protoni, neutroni e reazioni nucleari non sono di interesse della chimica, in quanto nelle reazioni chimiche il nucleo rimane intoccato, reazioni chimiche sono relazioni tra elettroni. 
\begin{itemize}
    \item \textbf{NUCLEONI} $\rightarrow$ Protoni e neutroni
    \item Somma del numero di protoni e neutroni, numero di massa $\mathcal{A}$
    \item Numero di protoni $\mathcal{Z}$
    \item Numero di neutroni $\mathcal{A-Z}$
\end{itemize}
\subsubsection*{Nota}
Specie chimiche con lo stesso numero di elettroni danno le stesse reazioni chimiche e possiedono le stesse proprietà chimico fisiche
\subsubsection{I NUCLIDI}
I nuclidi sono le diverse specie di atomi caratterizzate da nuclei di definita composizione Un nuclide viene rappresentato dal simbolo che indica la particella, con apice e pedice a sinistra per indicare numero di massa e numero atomico, apice e pedice a destra per indicare la carica e il numero di atomi
\[
^\mathcal{A}_\mathcal{Z}X^\pm _n
\]
Due nuclidi si dicono \textbf{isobari} se han diverso numero atomico ma ugual numero di massa
\[
^{14}_6C \qquad  ^{14}_7N
\]
Due nuclidi si dicono \textbf{isotopi} se han lo stesso numero atomico ma diverso numero di massa. Cambia il numero di neutroni
\[
^{16}_8O \qquad ^{17}_8O
\]
\subsubsection{Gli isotopi}
Tutti gli elementi chimici presenti in natura sono costituiti da miscele di \textbf{nuclidi diversi},
caratterizzati tutti dallo stesso numero atomico ma diverso numero di massa, isotopi appunto.\\
\[
^{20}_{10}Ne \qquad ^{21}_{10}Ne \qquad^{22}_{10}Ne 
\]
Esempi di isotopi di neon
\begin{equation}\label{abbondanza_ox}
^{16}_{8}O \ p= \ 99.979\% \qquad ^{17}_{8}O \ p= \ 0.037\%  \qquad^{18}_{8}O \ p= \ 0.204\%    
\end{equation}

Isotopi naturali dell'ossigeno, non radioattivi\\
Gli isotopi naturali sono presenti in natura in forma stabile, non sono soggetti a decadimento radioattivo.
La \textit{abbondanza percentuale} è intesa come numero di atomi, e sono costanti nell'universo, salvo ambienti specifici dove possono casualmente esserci variazioni minime dei valori
\[
^{1}_{1}H \ Prozio \ p= \ 99.984\% \qquad ^{2}_{1}H \ Deuterio\ p= \ 0.016\% 
 \qquad^{3}_{1}H \ Trizio \ p= \ 10^{-15}\% \]
Il prozio è notato anche semplicemente con Idrogeno, il deuterio con Idrogeno pesante e si nota anche solo con $D$, o $^2_1D$, Il Trizio con $T$ o $^3_1T$\\
Se noi sostituissimo all'idrogeno il deuterio per formare l'acqua, per formare l'acqua pesante $^2_1H_2O$, questo elemento bolle a $101.43C$, e fonde a $3.82C$.\\
\subsection{La massa}
\subsubsection{La massa atomica}
Le masse degli atomi vengono determinate con uno strumento che si chiama spettrometro di massa e sono masse atomiche \textbf{assolute},
i loro pesi sono molto piccoli.
E' stato perciò conveniente definire una particolare unità di misura, l'unità di massa atomica (uma), definita come \\
1 uma = $\frac{1}{12}$ della massa di $^{12}_6C$ = $1.6606 x 10^{-24}g$\\
Dividendo la masssa atomica assoluta per l'unità di massa atomica posso ottenere la massa atomica relativa.
\[m_{ar}= \frac{m_{aa}}{uma}\]
\subsubsection{La massa di un elemento}
\qquad La massa di un elemento, ovvero la massa atomica media o il peso atomico.\\
Abbiamo visto che gli elementi sono composti da miscele di diversi isotopi. Nei calcoli chimici
dobbiamo considerare la presenza di ogni isotopo per ottenere il valore medio della massa atomica di un 
elemento. Indicando con "E" un generico elemento della tavola periodica, avremo 
\begin{equation}\label{massa_atomica}M_e= \frac{\Sigma _i M_i \ p_i}{100}\end{equation}
$M_i$ è calcolato in uma.
Esempio:
\[M_{^{16}_8O}=15.9949 \ uma \]
\[M_{^{17}_8O}=16.9991 \ uma \]
\[M_{^{18}_8O}=17.9992 \ uma \]
Da cui, utilizzando la formula \eqref{massa_atomica} e conoscendo l'abbondanza percentuale \eqref{abbondanza_ox}, si ottiene la massa media dell'ossigeno
\[M_{O}=15.9994 \ uma \]
\subsubsection*{Nota}
Conoscendo la formula chimica e sommando le masse degli atomi si ottiene la massa molecolare.
\subsubsection{La mole ed il Numero di Avogadro}
La maggior parte delle reazioni chimiche coinvolge un numero enorme di particelle(atomi, molecole etc...).
E' stato perciò conveniente definire una nuova grandezza, la mole (mol), che rappresenta un numero grande e 
fisso di particelle, comparabile alla quantità di particelle usate in un esperimento reale.\\
DEF: \textit{Una mole di una determinata sostanza è la quantità di tale sostanza che contiene tante
entità elementari (Atomi, ioni, molecole etc...) quanti sono gli atomi contenuti in 12 g esatti di $^{12}_6C$.}\\ 
Possiamo determinare facilmente questo numero di atomi: basta dividere la massa totale per la massa di un atomo.
Da questo il numero di Avogadro è
\\ $N_A = 6.022\cdot 10^{23}$ entità/mol.\\
\subsubsection{La massa molare}
Esempio : \textit{Si voglia determinare la massa molare di $H_2O$}
\begin{enumerate}
    \item La massa espressa in uma di $H_2O$ è di 18.015 uma  
    \item La massa in grammi di una molecola di $H_2O$ è 18.015 uma  $\cdot 1.6606\cdot 10^{-24}$
    \item In una mole di $H_2O$ ci sono $N_A$ molecole
    \item La massa espressa in grammi di una mole di $H_2O$ è (numero molecole)$\cdot$(massa in g di una molecola)
\end{enumerate}
Una mole di $H_2O$ pesa 18.015 g, così per qualsiasi sostanza, il peso di una mole di un dato elemento X
è pari allo stesso valore numerico della sua massa molecolare o atomica espressa in uma.

\subsubsection*{Il numero di Moli}
Ne segue dalla definizione di massa molare che è possibile impostare la seguente relazione
\begin{equation*}
    M_{(g)} \ : \ 1_{(mol)} \ = \ m_{(g)} \ :\ n{(mol)} \\
    n \ = \frac{m}{M}
\end{equation*}
\subsubsection{Gli elementi e i composti}
Le sostanze pure presenti in natura o prodotte artificialmente si dividono in:
\begin{itemize}
    \item Sostanze elementari
    \item Composti
\end{itemize}
Le sostanze elementari sono formate da un solo tipo di elemento, presente con uno o pià atomi.\\
Le sostanze elementari si presentano sotto forma di 
\begin{enumerate}
    \item Singoli atomi isolati
    \item Atomi combinati in molecole distinte, come $H_2$
    \item Aggregati di atomi dove non è possibile individuare nessuna entità discreta, come una barretta di ferro
\end{enumerate}

I composti sono invece formati da elementi diversi. Si possono presentare sotto forma di 
\begin{enumerate}
    \item Molecole distinte, come il metano $CH_4$
    \item Insiemi di ioni aggregati da forze di natura elettrostatica, esempio $NaCl$ (Non esistono molecole
    ioniche, sono in realtà sviluppi tridimensionali compenetrati)
    \item Concatenzioni infinite di atomi, esempio $SiO_2$ . Non si trovano molecole singole di tali composti. sviluppo tridimensionale di $SiO_4^{2-}$
\end{enumerate}
\subsection{Il significato dei simboli e delle formule chimiche}
Le sostanze elementari vengono rappresentate con simboli. I composti vengono rappresentati con formule
Tanto i simboli quanto le formule hanno un duplice significato:
\begin{enumerate}
    \item Qualitativo : Indicano un certo tipo di sostanza
    \item Quantitativo: Rappresentano contemporaneamente un atomo o una molecola o una quantità in grammi pari ad una mole
\end{enumerate}
\subsubsection{Il numero di ossidazione}
Trattando il legame chimico vedremo che quano due o più atomi si legano insieme per formare un aggregato
poliatomico avviene sempre una redistribuzione degli elettroni, i quali vengono persi, acquistati, condivisi etc\dots \\
Il numero di Ossidazione $no$ è un parametro che tiene conto di questa nuova ridistribuzione.\\
Non è in realtà qualcosa che esiste nella realtà, traspone un concetto appartenente ai legami ionici, è stato inventato per due motivi:
\begin{enumerate}
    \item Nomenclatura in chimica inorganica
    \item Bilanciamento delle reazioni di ossidoriduzione complesse
\end{enumerate}
Il numero di ossidazione viene indicato ponendo sopra il simbolo di ogni elemento il suo $no$.\\
%\ch{\ox{+1,H2}\ox{-2,O}}
Il numero di ossidazione può essere determinato:
\begin{enumerate}
    \item Conoscendo la formula di struttura di LEWIS del composto e i valori di elettronegatività degli elementi presenti nel composto
    \item Facendo uso delle \textbf{REGOLE EMPIRICHE}
    \begin{enumerate}
        \item il no degli elementi nelle sostanze elementari è zero
        \item Gli elementi elencati presentano invariabilmente in tutti i loro composti il seguente no:
        \begin{itemize}
            \item Tutti i metalli alcalini (Prima colonna a sx) sempre +1
            \item Tutti i metalli alcalino-terrosi (Seconda colonna a sx) sempre +2
            \item Cadmio Cd sempre +2
            \item Zinco Zn sempre +2
            \item boro B sempre +3
            \item Alluminio Al sempre +3
            \item Fluoro F sempre -1
            \item Argento Ag sempre +1
            \item Idrogeno H sempre +1 tranne negli idruri metallici dove no = -1
            \item Ossigeno O sempre -2 tranne in OF$_2$ dove ha no pari a +2 e nei perossidi o superossidi
        \end{itemize}
        \item Tutti gli altri elementi della tavola periodica presentano no variabili da composto a composto.
        \begin{itemize}
            \item Ferro : +2 , +3
            \item Rame : +1 , +2
            \item Zolfo : -2 , +4 , +6
            \item Cloro, Bromo e Iodio : -1 , +1 , +3 , +5 , +6
        \end{itemize}
        \item La somma algebrica dei no di tutti fli atomi di ciascun elemento presenti nella formula è uguale alla carica della formula
        \[\Sigma no \ = \ \text{carica formula}\]
        Esempi: determinare no di $SO_4^{2-}$ (+6), no di $Cu^{2+}$ (+2), utilizzando la formula sopra.\\
    \end{enumerate}
\end{enumerate}
\subsubsection{La rappresentazione delle equazioni chimiche}
Le equazioni chimiche rappresentano i cambiamenti che avvengono nelle sostanze durante le reazioni chimiche. 
Tutte le reazioni chimiche si possono rappresentare con questi due simbologie:
\[aA \ + \ bB \rightarrow lL \ + \ mM\]
\[aA \ + \ bB \rightleftharpoons lL \ + \ mM\]
Dove le lettere maiuscole sono le formule chimiche, le lettere minuscole i coefficienti stechiometrici.
\subsection{La classificazione delle reazioni chimiche}
\begin{itemize}
    \item Sulla base della completezza :  
    \begin{itemize}
        \item Reazioni complete
        \item Reazioni di equilibrio
    \end{itemize}
    \item Dal tipo di reazione 
    \begin{itemize}
        \item Redox
        \item Non Redox
    \end{itemize}
    \item Su base di calore ceduto o acquistato
    \begin{itemize}
        \item Esotermiche
        \item Endotermiche
    \end{itemize}
\end{itemize}
\subsubsection{La classificazione sulla base della completezza}
\begin{enumerate}
    \item Reazione completa: Almeno un reagente si trasforma completamente nel prodotto
    
    \item Reazione di equilibrio: Ci sono reazioni nelle quali la reazione inizia, ed arrivati ad un certo punto detto di "equilibrio chimico",
    le quantità di prodotti che si formano non vengono più aumentate e neanche
    diminuite, perchè le ulteriori quantità di prodotto che si formano reagiscono tra di loro pre ridare i reagenti di partenza
    \[N_{2(g)}+3H_{2(g)}\ \rightleftharpoons 2NH_{2(g)}\]
    Questa reazione di equilibrio da 32 g di N2 produce solo 1 g di $NH_2$\\

\end{enumerate}
\subsubsection{Classificazione sulla base del tipo di reazione}
\begin{enumerate}
    \item Reazioni non Redox : Sono le reazioni che avvengono senza variazioni del numero di ossidazione
        degli elementi presenti
         
    \item Reazioni Redox : Sono le reazioni che avvengono con variazioni del numero di ossidazione degli
            elementi presenti.
\end{enumerate}
\subsubsection*{Le reazioni Redox}
Una specie chimica si dice che si ossida quando un elemento che la costituisce aumenta il suo no.     \\
Una specie chimica si dice che si riduce quando un elemento che la costituisce diminuisce il suo no. \\
Una sostanza è detta \textit{agente riducente} quando riduce un'altra sostanza e viene anche chiamata \textit{FORMA RIDOTTA (Rid)}
di una coppia redox.\\
Una sostanza è detta \textit{agente ossidante} quando ossida un'altra sostanza e viene anche chiamata \textit{FORMA OSSIDATA (Ox)}
di una coppia redox.\\
\[Rid_1 \rightleftharpoons Oss_1 + ne^-\]
Questa scrittura si chiama \textit{semireazione di ossidazione}. Gli elettroni persi devono essere acquistati da una specie due
\[Oss_2 + ne^- \rightleftharpoons Rid_2\]
Una reazione di ossidoriduzione è la somma di due semireazioni
\[Rid_1 + Oss_2 \rightleftharpoons Oss_1 + Rid_2\]
Esempio
\[Zn + Cu^{2+} \rightleftharpoons Zn^{2+} + Cu\]
\[Oss_1/Rid_1 e Oss_2/Rid_2 \text{costituiscono due coppie coniugate redox}\]
\subsection*{Le reazioni non Redox}
\textbf{Il concetto di acido-base secondo Bronsted e Lowry:}\\ \textit{Un acido è una specie donatrice di protoni $H^+$, mentre una base 
è una specie accettrice di protoni $H^+$}\\
\[Acido_1 \rightleftharpoons Base_1 + H^+ \ \text{semireazione acido-base}\]
\[Base_2 + H^+ \rightleftharpoons Acido_2 \ \text{semireazione acido-base}\]
\[Acido_1 + Base_2 \rightleftharpoons Base_1 + Acido_2 \ \text{reazione acido-base}\]
\subsubsection{Il bilanciamento delle reazioni chimiche}
Il processo di bilanciamento di qualsiasi reazione chimica si basa su due principi:
\begin{enumerate}
    \item Bilancio di massa : numero e tipo di atomi di ciascun elemento deve essere uguale nei due membri dell'equazione.
    \item Bilancio di carica : La somma algebrica delle cariche nei due membri dell'equazione deve coincidere
\end{enumerate}
Questi due principi vengono applicati contemporaneamente modificando gli oportuni coefficienti stechiometrici. Si usano quattro metodi:
\begin{enumerate}
    \item Metodo empirico (o metodo per verifica). E' utile per bilanciare reazioni non redox e reazioni redox semplici
    \item Metodo algebrico
    \item Metodo diretto
    \item Metodo delle semireazioni
\end{enumerate}
Noi adotteremo solo quello empirico, gli ultimi sono utili per reazioni complesse.\\
\textbf{Metodo empirico} significa aggiustare i coefficienti stechiometrici dati inizialmente per tentativi
fintantochè non si ottiene la reazione bilanciata correttamente.
Schema:
\begin{enumerate}
    \item Si bilancia un elemento alla volta
    \item I coefficienti stechiometrici ottenuti in ogni passaggio non vanno più modificati fino al termine del processo.
\end{enumerate}
Strategia:
\begin{enumerate}
    \item Se un elemento compare solo in un composto nelle due parti dell'equazione, si deve bilanciare per primo questo elemento; 
        Nel caso si presenti questa situazione per più di un elemento si inizierà il bilanciamento dall'elemento con numero di atomi diseguale.
    \item Quando uno dei reagenti o dei prodotti è presente come sostanza elementare, si debe bilanciare questo elemento per ultimo.
\end{enumerate}
\end{document}